%adobe reader fix
\pdfminorversion=4

\documentclass[final,12pt]{beamer}

%set image/logo options
% Rice
\def\RightLogoWidth{0.18}
\def\RightLogoPaddingTop{0.25cm}
\def\RightLogoPaddingBottom{0.5cm}
\def\RightLogo{../../logos/RiceLogo_TMCMYK300DPI.jpg}
% SHINE
\def\LeftLogoWidth{0.17}
\def\LeftLogoPaddingTop{0.5cm}
\def\LeftLogoPaddingBottom{0.5cm}
\def\LeftLogo{../../logos/shine-logo.png}
% SunPy
\def\AnotherLeftLogoWidth{0.06}
\def\AnotherLeftLogoPaddingTop{0.25cm}
\def\AnotherLeftLogoPaddingBottom{0.25cm}
\def\AnotherLeftLogo{../../logos/sunpy_powered_logo.png}
% Title and author(s) block
\def\TitleWidth{0.55}
% GitHub
\def\GitHubLogoWidth{0.014\paperwidth}
\def\GitHubLogo{../../logos/GitHub-Mark-120px-plus.png}
\def\GitHubUser{wtbarnes}
% Affiliation in footer
\def\AffiliationFooter{Department of Physics and Astronomy - Rice University - Houston, TX USA}
% Email
\def\EmailAddressFooter{will.t.barnes@rice.edu}

%set theme
\mode<presentation>
{
\usetheme{I6dv_custom}
}
\setbeamertemplate{caption}[numbered]
\usefonttheme[onlymath]{serif}

%Include packages
\usepackage{soul,color,verbatim}
\usepackage{amsmath,amsthm,amssymb}
\usepackage{graphicx}
\usepackage{epstopdf}
\usepackage[numbers]{natbib}
\usepackage{multicol}
\usepackage[english]{babel}
%\usepackage[latin1]{inputenc}
\usepackage{tikz}
%setup beamerposter package, use 3-by-4 feet
\usepackage[orientation=portrait,size=custom,width=91.44,height=121.92,scale=1.0]{beamer/beamerposter/beamerposter}

%tikz configuration

%custom commands go here
\newcommand{\ang}{\AA~} %alias angstrom
\setbeamerfont{caption}{size=\footnotesize} %make caption size small

%Set author and title
\title[]{Using Synthetic and Observed Timelags to Constrain Nanoflare\\ Heating Properties in Active Region Cores}
\author[Barnes, Bradshaw, \& Viall]{Will T. Barnes\inst{1}, Stephen J. Bradshaw\inst{1}, \& Nicholeen Viall\inst{2}}
\institute[]{\inst{1} Department of Physics and Astronomy, Rice University \inst{2} NASA Goddard Space Flight Center}
\date{30 July-3 August, 2018}

%start poster
%everything goes in one frame
\begin{document}
\begin{frame}
  %start columns environment to slice up the page horizontally
  \begin{columns}[T]
  \hfill
  %%
  %%first column
  \begin{column}{0.49\linewidth}
    %
    %introduction
    \begin{block}{Introduction}
    \begin{itemize}
      \item Intro goes here
    \end{itemize}
    \end{block}
    %
    % forward modeling
    %% Describe synthesizAR code, maybe a diagram?
    \begin{block}{Pipeline for Forward Modeling Emission from Active Region Cores}
    We have developed a Python package for forward modeling emission from ARs using ensembles of field-aligned hydrodynamic models. It leverages the full power of the scientific Python stack and relies heavily on the SunPy \citep{sunpy_community_sunpypython_2015} and Astropy \citep{astropy_collaboration_astropy:_2013} libraries.
    %
      \begin{columns}[T]
      \begin{column}{0.39\columnwidth}
      \end{column}
      \begin{column}{0.59\columnwidth}
        \begin{enumerate}
          \item Fetch observed magnetogram for the desired AR. Fig. \ref{fig:hmi_map_with_lines} shows an HMI magnetogram of NOAA 1109. 
          \item Perform a field extrapolation (e.g. PFSS) to derive the three-dimensional vector field $\vec{B}$
          \item Trace 1000 fieldlines through the extrapolated field, including only closed fieldlines in the range $10<L<1000$ Mm.
          \item For each fieldline, run a field-aligned hydrodynamic model. In this case, we'll use the two-fluid EBTEL model described in \citet{barnes_inference_2016}
          \item Map $T_e$ and $n_e$ from simulations to 3D field skeleton and calculate emissivity for each selected transition $\lambda_{ij}$ of element $\mathrm{X}$ and charge state $k$,
            \begin{equation*}
              \varepsilon_{ij}^{X,k} = n_jA_{ij}hc/\lambda_{ij}/n_e \quad [\mathrm{erg}\,\mathrm{cm}^{3}\,\mathrm{s}^{-1}]
            \end{equation*}
            where all atomic data comes from the CHIANTI atomic database \citep{young_chianti_2016,dere_chianti_1997}
          \item Integrate the emissivity along the LOS for each transition,
            \begin{equation*}
              I(\lambda_{ij}) = \frac{1}{4\pi}\int_{\mathrm{LOS}}\mathrm{d}h\,0.83\mathrm{Ab}(X)f_{X,k}\varepsilon_{ij}^{X,k}n_e^2 \quad [\mathrm{erg}\,\mathrm{cm}^{-2}\,\mathrm{s}^{-1}\,\mathrm{str}^{-1}]
            \end{equation*}
            \begin{itemize}
              \item $f_{X,k}$ calculation includes effects due to nonequilibrium ionization \citep[e.g.][]{bradshaw_numerical_2009,bradshaw_what_2011}
              \item $T_e,n_e$ functions of $h$, the distance along the LOS which intersects \textit{many} loops
            \end{itemize}
          \item Convolve with wavelength response function of each channel of each instrument. Here, we synthesize observations from both channels of the Extreme-ultraviolet Imaging Spectrometer (EIS) on \textit{Hinode}.
        \end{enumerate}
      \end{column}
      \end{columns}
    \end{block}
    %
    % Heating model
    %% Show tN distributions
    \begin{block}{Heating Model: Impulsive Heating over a Range of Frequencies}
      \begin{itemize}
        \item Describe heating model here
      \end{itemize}
    \end{block}
    %
    % Forward modeling
    %% How to compute AIA intensities
    \begin{block}{Forward Modeling AIA Intensities}
      \begin{itemize}
        \item Details about AIA calculation
        \item list ions/elements
        \item Show equations
      \end{itemize}
    \end{block}
    %
  \end{column}
  %%
%%%%%%%%%%%%%%%%%%%%%%%%%%%%%%%%%%%%%%%%%%%%%%%%%
  %%second column
  \begin{column}{0.49\linewidth}
    %
    % em distributions
    %% maps of true versus predicted EM for a all tN and a few T bins
    \begin{block}{Simulated AIA Intensities}
    \end{block}
    %
    % timelags
    %% quantitative and qualitative explanation
    \begin{block}{Computing Timelags Between AIA Channel Pairs}
    \end{block}
    %
    % simulated and observed timelags
    %% maps of timelags
    \begin{block}{Simulated versus Observed Timelags}
    \end{block}
    %
    % timelag classification
    %% distribution of slopes plus overplotted results from literature
    \begin{block}{Analyzing Observed Pixels with a Random Forest Classifier}
    \end{block}
    %
    %Conclusions
    \begin{block}{Conclusions}
      \begin{itemize}
        \item The conclusions go here
      \end{itemize}
    \end{block}
    %
    %references
    \begin{block}{References}
      \scriptsize
      \begin{multicols}{2}
        \bibliographystyle{../../apj.bst}
        \bibliography{references.bib}
      \end{multicols}
    \end{block}
  \end{column}
  \end{columns}
\end{frame}
\end{document}
