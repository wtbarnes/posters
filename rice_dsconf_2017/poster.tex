%adobe reader fix
\pdfminorversion=4

\documentclass[final]{beamer}

%set image/logo options
% Rice
\def\RightLogoWidth{0.18}
\def\RightLogoPaddingTop{0.25cm}
\def\RightLogoPaddingBottom{0.5cm}
\def\RightLogo{../logos/RiceLogo_TMCMYK300DPI.jpg}
% SunPy
\def\LeftLogoWidth{0.06}
\def\LeftLogoPaddingTop{0.25cm}
\def\LeftLogoPaddingBottom{0.25cm}
\def\LeftLogo{../logos/sunpy_powered_logo.png}
% Title and author(s) block
\def\TitleWidth{0.7}
% GitHub
\def\GitHubLogoWidth{0.014\paperwidth}
\def\GitHubLogo{../logos/GitHub-Mark-120px-plus.png}
\def\GitHubUser{wtbarnes}
% Affiliation in footer
\def\AffiliationFooter{Department of Physics and Astronomy - Rice University - Houston, TX USA}
% Email
\def\EmailAddressFooter{will.t.barnes@rice.edu}

%set theme
\mode<presentation>
{
\usetheme{I6dv_custom}
}
\setbeamertemplate{caption}[numbered]

%Include packages
\usepackage{soul,color,verbatim}
\usepackage{type1cm}
\usepackage{calc}
%\usepackage{times,mathptmx}
\usepackage{amsmath,amsthm,amssymb,latexsym}
\usepackage{empheq}
\usepackage{graphicx}
\usepackage{epstopdf}
\usepackage[numbers]{natbib}
\usepackage{multicol}
\usepackage{subfigure}
\usepackage[english]{babel}
%\usepackage[latin1]{inputenc}
\usepackage{tikz}
%setup beamerposter package
\usepackage[orientation=portrait,size=custom,width=91.44,height=121.92,scale=1.0]{beamer/beamerposter/beamerposter}

%tikz configuration

%custom commands go here
\newcommand{\ang}{\AA~} %alias angstrom
\setbeamerfont{caption}{size=\footnotesize} %make caption size small

%Set author and title
\title[AR Timelag Analysis]{Timelag Analysis of Global Hydrodynamic Simulations of Active Regions in the Solar Corona}
\author[Barnes \& Bradshaw]{Will T. Barnes \& Stephen J. Bradshaw}
\institute[Rice University]{Department of Physics and Astronomy, Rice University\\
                            Rice Data Science Conference, 9-10 October 2017}
\date{9-10 October, 2017}

%start poster
%everything goes in one frame
\begin{document}
\begin{frame}
  %start columns environment to slice up the page horizontally
  \begin{columns}[T]
  \hfill
  %%
  %%first column
  \begin{column}{0.49\linewidth}
    %
    %introduction
    \begin{block}{The Coronal Heating Problem}
    \begin{itemize}
      \item Fundamental question: \alert{what is the frequency of energy release in the cores of active regions (ARs)?}
      \item Define heating frequency in terms of $t_N$, the time between successive heating events on a \textit{single strand}, and $\tau_{cool}$, a typical loop cooling time,
      \begin{itemize}
        \item Low-frequency heating: $t_N>\tau_{cool}$ (i.e. approaches single nanoflare case), 
        \item High-frequency heating: $t_N<\tau_{cool}$ (i.e. approaches steady heating case), 
      \end{itemize}
      \item Using a robust forward modeling framework and emission measure diagnostics, we address two primary questions:
      \begin{enumerate}
        \item \alert{What are the observational signatures of nanoflares of varying frequency?}
        \item \alert{Can these signatures be observed and be used to constrain the heating frequency in AR cores?}
      \end{enumerate}
    \end{itemize}
    \end{block}
    %
    %% loop hydrodynamics
    %% Talk about hydro models
    \begin{block}{Loop Hydrodynamics}
    \begin{itemize}
        \item Loop hydro discussion
        \item Governing equations
        \item Sample curves
    \end{itemize}
    \end{block}
    %
    % forward modeling
    %% Describe synthesizAR code, maybe a diagram?
    \begin{block}{Forward Modeling Emission from Active Region Cores}
        % forward modeling details here
        % show HMI map with lines
        % equation for intensity calculation
        % discuss steps
    \end{block}
    %
    % Heating model
    %% Explain heating models, energy distribution
    \begin{block}{Heating Models}
      \begin{columns}[T]
      \begin{column}{0.49\columnwidth}
        \begin{itemize}
        \item Nanoflare model of \citet{parker_nanoflares_1988}: corona heated by impulsive ($\ll\tau_{cool}$), low-energy ($\sim10^{24}$ erg) events produced by twisting, braiding of field lines
        \item Each strand heated independently by repeating triangular pulses of duration $\tau=200$ s; preferentially heat electrons 
        \item Use extrapolated field strength to estimate total energy input per strand as,
          \begin{equation*}
            E = (\epsilon B)^2/8\pi
          \end{equation*}
        \item $\epsilon=0.1$ is the stressing coefficient and $B$ is the average field strength per strand
        \item Event energies chosen from a power-law distribution with $\alpha=-2.5$ such that total energy constrained as above
        \item Choose four different average waiting times $\langle t_N\rangle=250,\,750,\,2500,\,5000$ s 
        \begin{itemize}
          \item Range from high-frequency heating (250 s) to low-frequency heating (5000 s)
          \item $t_{N,i}\propto E_i$ such that larger events require a longer ``winding time'', consistent with Parker nanoflare picture \citep[e.g.][]{cargill_active_2014,barnes_inference_2016-1}
        \end{itemize}
        \end{itemize}
      \end{column}
      \begin{column}{0.49\columnwidth}
        \begin{figure}
        %\includegraphics[width=\columnwidth]{}
        \caption{Energy distribution} 
        \label{fig:wait_times}
        \end{figure}
      \end{column}
      \end{columns}
    \end{block}
    %
  \end{column}
  %%
%%%%%%%%%%%%%%%%%%%%%%%%%%%%%%%%%%%%%%%%%%%%%%%%%
  %%second column
  \begin{column}{0.49\linewidth}
    %
    % AIA intensity
    %% AIA intensity maps for a given heating model 
    \begin{block}{Synthesizing AIA Intensities}
      \begin{figure}
        %\includegraphics[width=\columnwidth]{}
        \caption{AIA intensities} 
        \label{fig:total_em_map}
      \end{figure}
    \end{block}
    %
    % 1d em distributions
    \begin{block}{Cross-correlation of Channel Pairs}
        % describe cross-correlation and timelag calculation
        % equations
        % 1D lightcurves plus correlations for cooling only
        % Dask dag for calculation + dashboard (?)
      \begin{figure}
        %\includegraphics[width=\columnwidth]{}
        \caption{caption goes here}
        \label{fig:em_1d}
      \end{figure}
    \end{block}
    %
    % timelag maps
    %% maps of maximum timelags
    \begin{block}{Timelag Maps}
      \begin{figure}
        %\includegraphics[width=\columnwidth]{}
        \caption{Timelag maps}
        \label{fig:em_slope_maps}
      \end{figure}
    \end{block}
    %
    %Conclusions
    \begin{block}{Conclusions}
      \begin{itemize}
        \item some Conclusions
        \item about the timelag results
        \item can go here
      \end{itemize}
    \end{block}
    %
    %references
    \begin{block}{References}
      \scriptsize
      \begin{multicols}{2}
        \bibliographystyle{../apj.bst}
        \bibliography{references.bib}
      \end{multicols}
    \end{block}
  \end{column}
  \end{columns}
\end{frame}
\end{document}
