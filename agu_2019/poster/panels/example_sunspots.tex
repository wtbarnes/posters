\begin{block}{Application: Automatic Detection of Sunspots in HMI Continuum Data}

\begin{itemize}
    \item Sunspot number is an important proxy for understanding solar activity and solar cycle variability
    \item Sunspot Tracking and Recognition Algorithm (STARA) code of \citet{watson_modelling_2009,watson_evolution_2011} automatically detects, tracks sunspots using top-hat transform and intensity thresholding
    \item Translated STARA into Python using scikit-image \citep{van_der_walt_scikit-image_2014} for image processing and \texttt{numpy} \citep{van_der_walt_numpy_2011} for array computation

\end{itemize}

% spell-checker: disable %
\begin{pycode}[manager]
# Load table
tab = astropy.table.Table.read(os.path.join(manager.data_dir, 'sunspot_catalogue.fits'))
tab['obstime'] = astropy.time.Time(tab['obstime'])
# Calculate monthly sunpot number
delta_time = tab['obstime'] - tab['obstime'][0]
bin_size = 30*u.day
spots,bins = np.histogram(
    delta_time.to(bin_size.unit).value,
    bins=np.arange(0,(delta_time[-1]+1*u.day).to(bin_size.unit).value, bin_size.value),)
time_spots = (bins[1:] + bins[:-1])/2*u.day + tab['obstime'][0]
kernel = Box1DKernel(6)  # smooth over 6 months because 1 time unit is 30 days
spots_smooth = convolve(spots, kernel)
# Plot
fig = plt.figure(figsize=texfigure.figsize(
    pytex,
    scale=1,
    height_ratio=0.4,
))
# sunspot number
ax = fig.add_subplot(121)
with time_support(simplify=True, format='decimalyear'):
    ax.plot(time_spots, spots, color=PALETTE[0], label='binned')
    ax.plot(time_spots, spots_smooth, color=PALETTE[1], label='smoothed')
ax.set_ylabel('Number of Spots (monthly total)');
ax.set_xlabel(f'Time');
ax.set_ylim(0, spots.max()*1.05);
ax.legend(loc=1)
# butterfly diagram
ax = fig.add_subplot(122)
with time_support(simplify=True, format='decimalyear'):
    ax.scatter(tab['obstime'],
               tab['center_lat'],
    marker='o', alpha=0.1, color=PALETTE[0])
ax.axhline(y=0, ls='--', color='k')
ax.axhline(y=-30, ls=':',color='k')
ax.axhline(y=+30, ls=':',color='k')
ax.set_ylim(-40,40)
ax.set_xlabel(f'Time')
ax.set_ylabel(f'Heliographic Latitude [{tab["center_lat"].unit}]');
plt.subplots_adjust(wspace=0.3)
# Save plot
tfig = manager.save_figure('sunspots', fext='.pgf')
tfig.caption = r'\textbf{Left:} Number of sunspots detected by STARA since the start of the SDO mission. The detections are binned into 30 day bins. The orange line shows the sunspot number smoothed over 6 months. \textbf{Right:} Heliographic Stonyhurst latitude (in degrees) of every detected sunspot as a function of time. The dotted lines denote $\pm\ang{30}$ and the dashed line denotes \ang{0}.'
tfig.figure_width = r'0.95\columnwidth'
\end{pycode}
\py[manager]|tfig|
% spell-checker: enable %

\begin{itemize}
    \item Applied STARA to 1 HMI continuum image per day from May 2010 to present ($>3400$ images)
    \item \alert{Using 660 cores across 55 compute nodes, built catalogue of all segmented regions in $<2$ minutes} (including reading data, resampling, and detection)
    \item Previously, $\approx24$ hours required to process 5000 daily MDI images--\alert{speedup by factor of $\sim500$}
    \item Such performance increases permit studies of how different algorithms, parameter selection may impact predictions of solar cycle variability
\end{itemize}
\end{block}