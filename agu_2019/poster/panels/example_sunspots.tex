\begin{block}{Example: Automatic Detection of Sunspots}

Describe STARA algorithm \citet{watson_modelling_2009}

What we did, why would we want to do this

How fast was it (compare to serial computation)? 

Read in data, resampled images, and performed automatic sunspot detection using STARA on 1 image per day for the entire mission and built catalogue of segmented regions in less than 2 minutes using 660 computing cores (55 nodes with 12 cores each)

Show sunspot count as function of time, butterfly diagram, maybe 

% spell-checker: disable %
\begin{pycode}[manager]
# Load table
tab = astropy.table.Table.read(os.path.join(manager.data_dir, 'sunspot_catalogue.fits'))
tab['obstime'] = astropy.time.Time(tab['obstime'])
# Calculate monthly sunpot number
delta_time = tab['obstime'] - tab['obstime'][0]
bin_size = 30*u.day
spots,bins = np.histogram(
    delta_time.to(bin_size.unit).value,
    bins=np.arange(0,(delta_time[-1]+1*u.day).to(bin_size.unit).value, bin_size.value),)
time_spots = (bins[1:] + bins[:-1])/2*u.day + tab['obstime'][0]
kernel = Box1DKernel(6)  # smooth over 6 months because 1 time unit is 30 days
spots_smooth = convolve(spots, kernel)
# Plot
fig = plt.figure(figsize=texfigure.figsize(
    pytex,
    scale=1,
    height_ratio=0.4,
))
# sunspot number
ax = fig.add_subplot(121)
with time_support(simplify=True, format='decimalyear'):
    ax.plot(time_spots, spots,)
    ax.plot(time_spots, spots_smooth)
ax.set_ylabel('Number of Spots (monthly total)');
ax.set_xlabel(f'Time');
ax.set_ylim(0, spots.max()*1.05);
# butterfly diagram
ax = fig.add_subplot(122)
with time_support(simplify=True, format='decimalyear'):
    ax.scatter(tab['obstime'][::10], tab['center_lat'][::10],
    marker='o', alpha=0.1)
ax.axhline(y=0, ls='--', color='k')
ax.axhline(y=-30, ls=':',color='k')
ax.axhline(y=+30, ls=':',color='k')
ax.set_ylim(-40,40)
ax.set_xlabel(f'Time')
ax.set_ylabel(f'Heliographic Latitude [{tab["center_lat"].unit}]');
plt.subplots_adjust(wspace=0.3)
# Save plot
tfig = manager.save_figure('sunspots', fext='.pgf')
tfig.caption = r''
tfig.figure_width = r'0.95\columnwidth'
\end{pycode}
\py[manager]|tfig|
% spell-checker: enable %

\end{block}