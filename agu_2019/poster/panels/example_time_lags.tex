\begin{block}{Application: Time Lag Analysis of Quiescent Active Regions}

\begin{itemize}
    \item Time lag analysis method of \citet{viall_evidence_2012} can provide \alert{important diagnostic of coronal heating models in active regions} \citep[e.g][]{barnes_understanding_2019}
    \item Time lag, $\tau_{AB}$, is the temporal delay that maximizes cross-correlation between light curves in channels $A$ and $B$,
    \begin{equation*}
        \tau_{AB} = \argmax_{\tau}\mathcal{C}_{AB},
    \end{equation*}
    where $\mathcal{C}_{AB}$ is the cross-correlation between channels $A$ and $B$,
    \begin{equation*}
        \mathcal{C}_{AB}(\tau) = \inversefourier{\fourier{\mathcal{I}_A(-t)}\fourier{\mathcal{I}_B(t)}},
    \end{equation*}
    and $\mathcal{I}_A(t)$ is the intensity in channel $A$ \citep{barnes_understanding_2019}
    \item $\tau_{AB}$ provides proxy for cooling time between channels, can reveal large-scale cooling patterns if calculated for each pixel in an active region
\end{itemize}

% spell-checker: disable %
\begin{pycode}[manager]
m_tl = sunpy.map.Map(os.path.join(manager.data_dir, 'time_lag_94_171.fits'))
m_cc = sunpy.map.Map(os.path.join(manager.data_dir, 'max_cc_94_171.fits'))
# Plot
fig = plt.figure(figsize=texfigure.figsize(
    pytex,
    scale=1,
    height_ratio=0.6,
    figure_width_context='figurewidth',
))
# timelag map
ax = fig.add_subplot(121, projection=m_tl)
im = m_tl.plot(
    axes=ax,
    vmin=-(6e3*u.s).to(u.s).value,
    vmax=(6e3*u.s).to(u.s).value,
    cmap='idl_bgry_004',
    title=False,
    annotate=False,
)
ax.grid(False)
ax.coords[0].set_axislabel('Helioprojective Longitude')
ax.coords[1].set_axislabel('Helioprojective Latitude')
ax.coords[1].set_ticklabel(rotation='vertical')
pos = ax.get_position().get_points()
cax = fig.add_axes([
    pos[0,0],
    pos[1,1]+0.01,  # pad a bit to lift off the top of the plot
    pos[1,0]-pos[0,0],  # width of the plot
    0.025,
])
cbar = fig.colorbar(im, cax=cax, orientation='horizontal')
#cbar.locator = matplotlib.ticker.FixedLocator([-1e2,0,1e2])
#cbar.update_ticks()
cbar.ax.xaxis.set_ticks_position('top')
# maxcc map
ax = fig.add_subplot(122, projection=m_cc)
im = m_cc.plot(
    axes=ax,
    vmin=0,
    vmax=1,
    cmap='magma',
    title=False,
    annotate=False,
)
ax.grid(False)
ax.coords[0].set_axislabel('Helioprojective Longitude')
ax.coords[1].set_axislabel(' ')
ax.coords[1].set_ticklabel(rotation='vertical',)
pos = ax.get_position().get_points()
cax = fig.add_axes([
    pos[0,0],
    pos[1,1]+0.01,  # pad a bit to lift off the top of the plot
    pos[1,0]-pos[0,0],  # width of the plot
    0.025,
])
cbar = fig.colorbar(im, cax=cax, orientation='horizontal')
#cbar.locator = matplotlib.ticker.FixedLocator([-1e2,0,1e2])
#cbar.update_ticks()
cbar.ax.xaxis.set_ticks_position('top')
# Save
tfig = manager.save_figure('time-lag-maps', fext='.pdf')
tfig.caption = r'Time lag (\textbf{left}) and maximum cross-correlation (\textbf{right}) maps of NOAA 11109 for the 94,171 \angstrom{} channel pair. The red and yellow colors in the left panel indicate plasma cooling from the 94 \angstrom{} channel into the 171 \angstrom{} channel.'
tfig.figure_width = r'0.75\columnwidth'
\end{pycode}
\py[manager]|tfig|
% spell-checker: enable %

\begin{itemize}
    \item Calculation parallelizes trivially using Fourier transform function provided by \texttt{dask} 
    \item For 12 hours of full cadence data cropped to a field of view $450''\times450''$, \alert{can compute $\tau_{AB}$ in $<30$ s using 1200 cores}
    \item Performing equivalent calculation in IDL by calling \texttt{c\_correlate.pro} in serial would take $\approx1$ hour -- \alert{speedup by a factor of $>100$}
\end{itemize}

% spell-checker: disable %
\begin{pycode}[manager]
tab = astropy.table.QTable.read(
    os.path.join(manager.data_dir, 'scaling_table.fits')
)
i_max = tab['compute_time'].argmax()
# Crude model of theoretical scaling
ideal_scaling = lambda x: tab['cores'][i_max] * tab['compute_time'][i_max] / x
# Plot
fig = plt.figure(figsize=texfigure.figsize(
    pytex,
    scale=0.65,
    height_ratio=0.6,
    #figure_width_context='figurewidth',
))
ax = fig.gca()
ax.plot(tab['cores'], tab['compute_time'],
        ls='', marker='o', markersize=12, color=PALETTE[0], label='measured')
ax.plot(tab['cores'], ideal_scaling(tab['cores']),
        color=PALETTE[1], label='ideal')
ax.set_yscale('log')
ax.set_xscale('log')
ax.set_xlabel('Number of Cores')
ax.set_ylabel(r'Compute Time [s]')
ax.legend(loc=1)
# Save
tfig = manager.save_figure('time-lag-scaling', fext='.pgf')
tfig.caption = r'Compute time of above time lag calculation as a function of number of computing cores. The blue dots show the measured compute times on the Merope cluster while the orange line shows the ideal scaling. The deviation from the ideal scaling is likely due to latency incurred by communication between workers.'
tfig.figure_width = r'\columnwidth'
\end{pycode}
\py[manager]|tfig|
% spell-checker: enable %

\end{block}